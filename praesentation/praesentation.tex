\documentclass[accentcolor=tud7b,noresetcounter]{tudbeamer}

\usepackage[ngerman]{babel}
\usepackage[utf8]{inputenc}
\usepackage[T1]{fontenc}

% footmisc behebt u.a. Probleme mit Fu?noten in Abschnittstiteln
\usepackage[stable]{footmisc}

% Einbinden von Grafiken erm?glichen
\usepackage{graphicx}

% Paket xtab erm?glicht Umbrechen von langen Tabellen
\usepackage{xtab}
\usepackage{tikz}

% picins erlaubt das Umflie?en von Abbildungen durch Text
% Untenstehendes renewcommand behebt den picins-bug, dass Abbildungen
% nicht im Abbildungsverzeichnis auftauchen
%\usepackage{picins}
\makeatletter
%\renewcommand\piccaption{\@dblarg{\@piccaption}}
\makeatother

\usepackage{verbatim}

% Paket setspace erlaubt Umschalten auf 1.5fachen Zeilenabstand
\usepackage{setspace}

\usepackage{hyperref}
\usepackage{listings}
\usepackage{amssymb}
\usepackage{newclude}
\usepackage{multirow}
\usepackage{array}
\usepackage{tabularx}
\usepackage{hyperref} 
\usepackage{graphicx}
\usepackage{pgfplots}
\usepackage{csvsimple}
%Erm?glicht Hyperlinks zwischen Textstellen und zu externen Dokumenten
%% breaklinks=true/false: Gibt an, ob Links umgebrochen werden d?rfen.
%% linktocpage=true/false: im Inhaltsverzeichnis sind nur die Seitenzahlen
%% links, nicht der Text
%% colorlinks=true/false: Links werden eingef?rbt (Farben werden mit
%% linkcolor, anchorcolor \dots festgelegt)
%% linkcolor=Farbe: Farbe des verlinkten Textes, Dokument-interne Links
%% citecolor=Farbe: Farbe des verlinkten Textes, Links zum
%% Literaturverzeichnis
%% filecolor=Farbe: Farbe des verlinkten Textes, Links auf lokale Dateien
%% urlcolor=Farbe: Farbe des verlinkten Textes, externe URLs
%% frenchlinks=true/false: Links werden als smallcaps, anstatt farbig
%% dargestellt.
%% breaklinks=true/false: Gibt an, ob Links umgebrochen werden d?rfen.
\hypersetup{%
  linktocpage=true,
  breaklinks=true,
  colorlinks=true,
  citecolor=black,
  urlcolor=black,
  linkcolor=black,
  pdfpagemode=UseThumbs,
  pdftitle=Übung 2 - Webmining,
  pdfauthor=Ingo Adrian und Steffen Pegenau,
  pdfsubject=Webmining,
  %pdfkeywords=xy
}


\lstset{ %
  %backgroundcolor=\color{white},   % choose the background color; you must add 
%\usepackage{color} or \usepackage{xcolor}
  %basicstyle=\footnotesize,        % the size of the fonts that are used for 
					%the code
  %breakatwhitespace=false,         % sets if automatic breaks should only 
%happen at whitespace
  breaklines=true,                 % sets automatic line breaking
  %captionpos=b,                    % sets the caption-position to bottom
  commentstyle=\bf,    % comment style
  %deletekeywords={...},            % if you want to delete keywords from the 
%given language
  %escapeinside={\%*}{*)},          % if you want to add LaTeX within your code
  %extendedchars=true,              % lets you use non-ASCII characters; for 
%8-bits encodings only, does not work with UTF-8
  frame=single,                    % adds a frame around the code
  %keepspaces=true,                 % keeps spaces in text, useful for keeping 
%indentation of code (possibly needs columns=flexible)
  %keywordstyle=\color{blue},       % keyword style
  language=html,                 % the language of the code
  %otherkeywords={*,...},            % if you want to add more keywords to the 
%set
  numbers=left,                    % where to put the line-numbers; possible 
%values are (none, left, right)
  %numbersep=5pt,                   % how far the line-numbers are from the code
  %numberstyle=\tiny\color{mygray}, % the style that is used for the 
%line-numbers
  %rulecolor=\color{black},         % if not set, the frame-color may be changed 
%on line-breaks within not-black text (e.g. comments (green here))
  %showspaces=false,                % show spaces everywhere adding particular 
%underscores; it overrides 'showstringspaces'
  %showstringspaces=false,          % underline spaces within strings only
  %showtabs=false,                  % show tabs within strings adding particular 
%underscores
  %stepnumber=2,                    % the step between two line-numbers. If it's 
%1, each line will be numbered
  %stringstyle=\color{mymauve},     % string literal style
  tabsize=4,                       % sets default tabsize to 2 spaces
  %title=\lstname                   % show the filename of files included with 
%\lstinputlisting; also try caption instead of title
}

% bibtex
%\usepackage[backend=biblatex]{biblatex}

%\bibliography{refs}
%\addbibresource{refs.bib}

%%%%%%%%%%%%%%%%%%%%%%%%%%%%%%%%%%%%%%%%%%%%%%%%%%%%%%%%%%%%%%%%%%%%%%%%%%%%%%%%%%%%%%

\title%[Klima \& Geographie] % (optional, only for long titles)
{Web Mining: Übung 3}
\subtitle{Lösungsvorschlag}

\author[Ingo Adrian und Steffen Pegenau]{}
\institute[Fachbereich Informatik]{}
\date[\today]
%%%%%%%%%%%%%%%%%%%%%%%%%%%%%%%%%%%%%%%%%%%%%%%%%%%%%%%%%%%%%%%%%%%%%%%%%%%%%%%%%%%%%

\pgfplotsset{compat=1.12}

\begin{document}
  
  \begin{titleframe}
  \end{titleframe}
  
  \begin{frame}
  \frametitle{Aufgabe 1:\\ Einteilung der Testdaten}
  \begin{itemize}
    \item Verwendeter Testdatensatz: Co-training Experiments for COLT 98\\
    \url{http://www.cs.cmu.edu/afs/cs.cmu.edu/project/theo-51/www/co-training/data/}
    \item Testdaten sind in Unterordnern sortiert:\\
    	  Art/Kategorie/VolleURL
   \item Art kann "`fulltext"' oder "`inlinks"'
   \item "`fulltext"': HTML-Dateien
   \item "`inlinks"': Link-Text(e), die auf die Seiten verwiesen haben. \\
	Beispiel: \texttt{<a href=VolleURL>Link-Text</a>}
   \item Zur einfacheren Verarbeitung soll die Ordnerhierarchie entfallen und
   	die Informationen in den Dateinamen übernommen werden.
  \end{itemize}
  \end{frame}
  
  \begin{frame}
  \frametitle{Aufgabe 1:\\ BASH: Neusortierung/Entfernung der Hierarchie \\ Teil 1: Die dunklen Zeichen }
  \begin{itemize}
    \item Mit einem Bash-Skript soll die Hierarchie flach gezogen werden
    \item Das Namensschema der Dateien soll lauten:\\
    $\text{fulltext|inlinks}\underbrace{-----}_{\text{Trenner}}\text{Kategorie}\underbrace{-----}_{\text{Trenner}}\text{VolleURL}$
    \item Die URL soll dabei außerdem angepasst werden:
	\begin{itemize}
		\item Doppelpunkte werden entfernt, weil sich Windows daran stört
		\item \texttt{http://www.} wird ersetzt, um die URLs zu verkürzen. Windows stört sich auch an langen Dateipfaden
	\end{itemize}
    \item Am Ende sehen die Dateinamen so aus: \\
    $\underbrace{fulltext}_{Art}--\underbrace{course}_{Kategorie}--\underbrace{BEG}_{http...}--\text{cs.washington.edu\^{}education\^{}courses\^{}431\^{}}$
    \item Vorteil des Ersetzens gegenüber dem Wegwerfen einzelner Zeichen/Abschnitte: Es gehen keine Informationen verloren. Bei Bedarf kann der gesamte Link wiederhergestellt werden.
  \end{itemize}
  \end{frame}
 
	\begin{frame}
		\frametitle{Aufgabe 1: \\ BASH: Neusortierung/Entfernung der Hierarchie \\ Teil 2: Angriff der Kategorisierung}
		\begin{itemize}
			\item Die Daten müssen nun noch in Test- und Trainingsdaten eingeteilt werden
			\item Das Skript wirft für jede Datei die Münze und dokumentiert das Ergebnis ebenfalls im Dateinamen
			\item Beispiel: \\
			\texttt{test----fulltext----course----BEG----cs.wisc.edu\^{}\~{}dyer\^{}cs766.html}
			\item Das Ergebnis der Verteilung scheint zufällig und damit brauchbar: \\
			\begin{tabular}{|l|r|r||r|}
				\hline
& Training	& Test	& 	Summe \\
\hline
course	& 	233	& 	227	& 	460	\\
non-course	& 	802	& 	840	& 	1642	\\
\hline
\hline
Summe	& 	1035	& 	1067	& 2102\\
				\hline
			\end{tabular}


		\end{itemize}

	\end{frame}


% etc
\end{document}